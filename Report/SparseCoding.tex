\documentclass[a4paper,10pt]{article}

\usepackage[utf8]{inputenc}
%\usepackage[T1]{fontenc}
%\usepackage[french]{babel}
\usepackage[margin=1in]{geometry} 
\usepackage{amsmath,amsthm,amssymb}
\usepackage{lmodern}
\usepackage{graphicx}
\usepackage{mathtools}
\usepackage{tcolorbox}
\usepackage{listings}
\usepackage{subcaption} % pour le subplot

\newcommand{\R}{\mathbb{R}}  
\newcommand{\Z}{\mathbb{Z}}
\newcommand{\N}{\mathbb{N}}
\newcommand{\Q}{\mathbb{Q}}
\newcommand{\C}{\mathbb{C}}
\DeclareMathOperator*{\argmax}{arg\,max}
\DeclareMathOperator*{\argmin}{arg\,min}
%opening
\title{Intership - Sparse Coding and Dictionary learning}
\author{Thomas Rolland}
\date{}%Remove date

\begin{document}


\begin{titlepage} % Suppresses displaying the page number on the title page and the subsequent page counts as page 1
	\newcommand{\HRule}{\rule{\linewidth}{0.5mm}} % Defines a new command for horizontal lines, change thickness here
		%------------------------------------------------
	%	Logo
	%------------------------------------------------
	
	\includegraphics[scale=2]{siteon0.png} % Include a department/university logo - this will require the graphicx package
\hfill
 \includegraphics[scale=2]{SAMOVA.png}
 % SAMOVA.png: 226x110 px, 300dpi, 1.91x0.93 cm, bb=0 0 54 26

	\center % Centre everything on the page
	
	%------------------------------------------------
	%	Headings
	%------------------------------------------------
	
	\textsc{\LARGE Université de Toulouse 3 - Paul Sabatier}\\[1.5cm] % Main heading such as the name of your university/college
	
	\textsc{\Large Institut de Recherche en Informatique de Toulouse - IRIT}\\[0.5cm] % Major heading such as course name
	
	\textsc{\large SAMOVA}\\[0.5cm] % Minor heading such as course title
	
	%------------------------------------------------
	%	Title
	%------------------------------------------------
	
	\HRule\\[0.4cm]
	
	{\huge\bfseries Neural network vs. statistical methods
\\ \vspace{0.5cm}- \vspace{0.5cm} \\  Comparison of methods for learning sound units using Dictionary Learning and Sparse Coding}\\[0.4cm] % Title of your document
	
	\HRule\\[1.5cm]
	
	%------------------------------------------------
	%	Author(s)
	%------------------------------------------------
	
	\begin{minipage}{0.4\textwidth}
		\begin{flushleft}
			\large
			\textit{Author}\\
			Thomas \textsc{Rolland} % Your name
		\end{flushleft}
	\end{minipage}
	~
	\begin{minipage}{0.4\textwidth}
		\begin{flushright}
			\large
			\textit{Supervisor}\\
			Thomas \textsc{Pellegrini}\\ % Supervisor's name\\
            Carine \textsc{Jauberthie}
		\end{flushright}
	\end{minipage}
	
	% If you don't want a supervisor, uncomment the two lines below and comment the code above
	%{\large\textit{Author}}\\
	%John \textsc{Smith} % Your name
	
	%------------------------------------------------
	%	Date
	%------------------------------------------------
	
	\vfill\vfill\vfill % Position the date 3/4 down the remaining page
	
	{\large\today} % Date, change the \today to a set date if you want to be precise
	

	%----------------------------------------------------------------------------------------
	
	\vfill % Push the date up 1/4 of the remaining page
	
\end{titlepage}

\begin{abstract}
Little document to summarize state-of-the-art sparse coding and dictionary learning for weekly supervised and unsupervised feature extraction in speech. Introduction to 1.2 is based on the Hugo Larochelle courses.\\
\begin{center}

\textbf{This document is a draft }
 
\end{center}
\end{abstract}


\tableofcontents


\section{Sparse coding}
%\documentclass[SparseCoding.tex]{subfiles}
\section{\textit{Sparseland}}

Modeling data play a central role in signal processing and machine learning. Sparse representation model also refers as \textit{Sparseland} \cite{8398588},  suggests a description of every signal as sparse linear combinations of basic elements called \textit{atoms}, which come from a redundant matrix called a \textit{dictionary}. Redundant matrix means that we have more atoms than the size of the corresponding signal. \\
The Sparseland model became one of the most important modelizations in image processing, and more generally in signal processing. Its give state-of-the-art results in a wide variety of tasks: Denoising, inpainting, image separation, deblurring,\dots\\
There is one fundamental property on which this theory is based, the same property which allows us to compress a signal, to denoise, enhance a degraded signal, \dots. All meaningful data sources (i.e signal) are structured.\\
Each source of information: image, audio, video, 3D object (meshes or points cloud),  text, financial time series, data on a graph,\dots  have an inner structure that is unique to them which can be characterized in various ways and allow redundancy.\\

The first signs of \textit{Sparseland} appear in the 1990s with the greedy and relaxed pursuit algorithm \cite{258082,305222} and the introduction of dictionary learning \cite{Olshausen97sparsecoding}.

\subsection*{The idea behind \textit{Sparseland}}
The first step in \textit{Sparseland} is to find (i.e learn) a dictionary with a relevant set of atoms (this step is called \textbf{Dictionary Learning}). Then the model assumption is that every incoming signal could be represented as a linear combination of only a few atoms from the dictionary (this step is called \textbf{Sparse Coding}).\\ \vspace{-0.4cm}
\begin{center}
\fbox{\begin{varwidth}{\textwidth}
\centering
\hspace{-5.5cm}\textbf{Definition:} Sparse \\ \vspace{0.2cm} \textit{ Occurring at widely spaced intervals, not thick or dense.}
\end{varwidth}}
%\fbox{\textbf{Definition:} \\Sparse:
%\textit{ Occurring at widely spaced intervals, not thick or dense.}}
\end{center}\vspace{0.4cm}
Unlike decomposition based on principal component analysis, \textit{Sparseland} does not impose that learned representation to be orthogonal, allowing more flexibility to adapt the representation to the signal.
\\
The idea of finding new representations of an input signal is not new. Signal processing's scientists already use this approach with predefined dictionaries instead of a learned one, for example, wavelets transform or Fourier transform. With a learned dictionary, we can expect to get more relevant information about the inner structure of the signal. For that reason, we can expect better performances in a wide range of applications.

\section{Mathematical formulation}


 \begin{figure}[h]
 \centering
 \includegraphics[scale=0.5]{spacecoding_12.png}
 % spacecoding_1.png: 943x256 px, 96dpi, 24.95x6.77 cm, bb=0 0 707 192
 \caption{Example of Sparse Coding reconstruction}
 \label{fig:exemple}
\end{figure}

Consider a signal $x_i \in \R^m$ in a set of signal $X = [x_1, x_2, \ldots, x_n] \in \R^{m \times n}$ ( generally $n$ is large and $m$ is small: $n >> m$), the aim of this method is to find a linear combination of overcomplete basis elements $D = [d_1, d_2, \ldots, d_k]$ under sparsity constraints which reconstruct the input signal (see  an example at figure \ref{fig:exemple}), overcomplete dictionary mean that $k > m$:
\begin{center}
 $x_i = D \gamma_i$
 \\ With $\gamma_i$ the sparse coefficients of the sparse decomposition for the signal $x_i$
\end{center}

The traditional way to get $\gamma_i$ is to optimize the empirical cost function.
\begin{center}
 $f_n(D) = \frac{1}{n} \sum_{i=1}^{n} l (x_i,D)$
\end{center}
Where $l$ is the cost function which is small where D is good for representing the signal. Usually, we use :
\begin{center}
 $l(x_i,D) =   \min\limits_{\gamma_i} \frac{1}{2} \underset{Squared\ error}{\underbrace{\| x_i - D \hspace{3px} \gamma_i \|_2^2}} + \lambda \underset{Sparsity\ term}{\underbrace{ \|\gamma_i \|_0}}$
\end{center}
Note that the cost function has two terms, which fight each other.  The first term corresponds to the squared error, while that the second term corresponding to the sparsity of the weight vector. $\lambda$ is a positive constant which controls the importance of the sparse term relative to the square error term. \\
The norm of all columns of D must be equal or less than 1  , otherwise, D could grow big while $\gamma$ becomes small to satisfy the prior.\\
This problem is also known as a \textit{basic pursuit}  or the  \textit{Lasso}. However, it is important to notice that the problem isn't convex and his optimization is NP-complete (due to the $l_0$ norm). Fortunately, it is well known that for kind of problem use $l_1$ norm instead of $l_0$ norm yields a sparse solution for $\gamma_i$.\\
Then the optimization problem can be rewritten as:
\begin{center}
 $\min\limits_{D} \frac{1}{n} \sum_{i=1}^{n}  \min\limits_{\gamma_i} \frac{1}{2} \| x_i - D \hspace{3px} \gamma_i \|_2^2 + \lambda \|\gamma_i \|_1$
\end{center}
Before presenting our experimentations we will explain how traditional Sparse Coding works. In particular, how this method learns redundant properties and patterns in a signal set. This is the learning phase.
\newpage
\section{Learning step}
A natural approach to solve this optimization problem is to alternate between the optimization of $D$ and $\gamma$, minimize over one while keeping the other one fixed.\\
\begin{algorithm}
 \label{algo:algo1}

 \caption{Learning step}
 \begin{algorithmic}
 \REQUIRE X the input signal
    \STATE $D_0$ initilized randomly, $\gamma$  is a zeros matrix
    \WHILE{D and $\gamma$ not converged}
        \STATE Fix D
        \STATE Find $\gamma$ \hspace{0.7cm} \textit{/* Sparse Coding step */}   (1)
        \STATE Fix $\gamma$
        \STATE Find D \hspace{0.7cm} \textit{/*Dictionary Learning  */} (2)
    \ENDWHILE
 \end{algorithmic}

\end{algorithm}

% 
% \paragraph{Idea} Sparse dictionary learning is a representation learning method which aims at finding a sparse representation of the input data (in form of a linear combination of basic elements (called Atoms). The idea of using a learned dictionary instead of a predefined one is based on wavelets.  The sparse learned models has recently led to state-of-the-art result for denoising, classification,...
% %UNSPERVISED LEARNING=========================================================+
% \paragraph{Unsupervised learning} Only use the inputs $x^{(t)}$ ( $X = [x_1,....,x_n]$ in $\R^{m \times n}$) for learning. Automatically extract meaningful features of our data, leverage the availability of unlabeled data and add a data-dependent regularize to trainings.\\
% 
% Sparse coding is one of the methods used for unsupervised learning  (like restricted Boltzmann machines and autoencoders).\\
% The idea behind sparse coding is: For each $x^{t}$ find a latent representation $\gamma^{t}$ such that:
% \begin{itemize}
%  \item[$\bullet$] It is sparse: the vector $\gamma^{t}$ has many zeros (only few nonzero elements)
%  \item[$\bullet$] We can reconstruct the original input $x^{(t)}$ as well as possible.
% \end{itemize}
% That mean, more formally:\\
% 
% %Formulation du problème=============================
% \begin{center}
%  $\min\limits_{D} \frac{1}{T} \sum_{t=1}^{T}  \min\limits_{\gamma^{(t)}} \frac{1}{2} \| x^{(t)} - D \hspace{3px} \gamma^{(t)} \|_2^2 + \lambda \|\gamma^{(t)} \|_1$\\
% \end{center}
% 
%  %Explicaiton de la formulation=========================
%  \begin{itemize}
%  \item[$\bullet$] D is a matrix of weights, usually refer to that matrix as a dictionary matrix (containt atoms) with $D \in  \R^{m \times k}$ ( k the number of atoms)
%   \item[$\bullet$] $\| x^{(t)} - D \hspace{3px} \gamma^{(t)} \|_2^2 $ is the reconstruction error
%   \item[$\bullet$]$ D \hspace{3px} \gamma^{(t)}$ is the reconstruction of $\hat{x}^{(t)}$
%   \item[$\bullet$]$\|\gamma^{(t)} \|_1$ is the sparsity penalty (more 0 in h we have, better it is)
%  \end{itemize}
% This two objectives fight each other. But it still a optimization problem (cf min), and we'll try to optimize it for each training example $x^{(t)}$. This is why we have a sum over all the training examples.
% \newline
% 
% %Contraintes sur D======================================
% \indent We also constrain the columns of D to be of norm 1 (otherwise, D could grow big while $\gamma^{(t)}$ becomes small to satisfy the prior). And sometimes the columns are constrained to be no greather than 1.\\
% 
% However, $\gamma^{(t)}$ is now a complicated function of $x^{(t)}$:\\
% Encoder is the minimization $\gamma(x^{(t)}) = arg\min\limits_{\gamma^{(t)}}= \frac{1}{2} \| x^{(t)} - D \hspace{3px} \gamma^{(t)} \|_2^2 + \lambda \|\gamma^{(t)} \|_1$, so the optimization problem is more complicated than a simple non linear problem.\\
% The idea to solve this minimization problem is \cite{NIPS2006_2979}
% 
% \begin{lstlisting}[language=Python,frame=single]
% while D not_converged :
%     Fix D
%     Minimize alpha     (1) // Sparse Coding step
%     Fix alpha
%     Minimize D           (2) // Update Dictionary step
% \end{lstlisting}
% 
% But there are some improvement like K-SVD algorithm \cite{KSVD}, which compute column by column a SVD computation over the relevant examples.
% %DICTIONARY==================================================================+  
% \paragraph{Dictionary}
% We can also write $\hat{x}^{(t)} = D\hspace{3px} \gamma(x^{(t)}) \displaystyle\sum_{\substack{k s.t.\\ \gamma(x^{(t)})_k\neq 0}}  D.,_k \gamma(x^{(t)} )_k$
% \begin{figure}[h]
%  \centering
%  \includegraphics[scale=0.5]{spacecoding_1.png}
%  % spacecoding_1.png: 943x256 px, 96dpi, 24.95x6.77 cm, bb=0 0 707 192
%   \caption{Example of reconstruction using sparse coding}
% \end{figure}
% 
% The images refer to $D.,_k$ (columns of D wich are not equals to 0) and the factor (1 or 0.8 in this case) refer to $\gamma(x^{(t)} )_k$
% \\We also refer to D as the dictionary:
% \begin{itemize}
%  \item[$\bullet$]in certain applications, we know what dictionary matrix to use
%  \item[$\bullet$]often however, we have to learn it
% \end{itemize}
% 
% In general we have $k<<n$ . But we can use an overcomplete dictionary with $k > m$.
% 
%\end{document}

\subsection{Inference of Sparse code}

We will focus here on the $\gamma$ calculation during the sparse coding step of learning algorithm (i.e. part (1) of algorithm 1). There are different ways to obtain the sparse coefficients: Some based on the $l_0$ norm, other on the $l_1$ norm.

\subsubsection{$l_0$ norm based}
Matching pursuit algorithm proposed in \cite{258082} intended to approximately solve the basic pursuit problem which is normally unacceptable in terms of calculation (if $D$ is large):
\begin{center}
 $\min\limits_{\gamma_i} \|\underset{r}{\underbrace{ x_i - D \hspace{3px} \gamma_i }}\|_2^2 \ s.t. \ \|\gamma_i \|_0 \leq L$\\
 (with $r$ the residual)
\end{center}
Matching pursuit algorithm is a greedy algorithm which iteratively generates a sorted list of atom and weighting scalars that represent the sub-optimal solution. An improvement of this algorithm is called Orthogonal Matching Pursuit which Orthogonalizes all the chosen elements.
\begin{algorithm}
 \caption{Orthogonal Matching Pursuit Algorithm}
\begin{algorithmic} 
 \REQUIRE D, $x$
 \STATE $\Gamma = \O$
 \WHILE{$\|\gamma\|_0 < L$}
    \STATE \textit{/* Pick the element that most reduces the objective */}
    \STATE $e \leftarrow \underset{i \in \Gamma}{\argmin} \{ \underset{\gamma}{\min}\|x - D_{\Gamma \cup \{e\}} \gamma\|_2^2\}$ 
    \STATE \textit{/* Update the active set */}
    \STATE $\Gamma \leftarrow \Gamma \cup \{e\}$
    \STATE \textit{ /* Update $\gamma$ and the residual */}
    \STATE $\gamma \leftarrow (D_\Gamma^{\intercal} D_\Gamma)^{-1} D_\Gamma^{\intercal} x$
    \STATE $r \leftarrow x - D\gamma_\Gamma$
 \ENDWHILE
\end{algorithmic}
\end{algorithm}


\subsubsection{$l_1$ norm based}
With this relaxation of the $l_0$ norm, we could use a gradient descent method to solve this optimization problem:\\
\begin{center}
$\Delta_{\gamma} l(x^{(t)}) = D^{\intercal} (D \gamma - x) + \lambda sign(\gamma)$
\end{center}
However,  it's well known that $l_1$ norm is not differentiable at 0, in this case, if $\gamma$ changes sign because of $l_1$ norm gradient then clamp to 0:
\begin{center}
 Gradient descent step for each $k_th$ element of $\gamma$:\\\vspace{0.2cm}
 $\gamma_k = \gamma_k - \alpha (D_{.,k})^{\intercal} (D \gamma - x)$\\\vspace{0.3cm}
 Clamp to 0:\\ \vspace{0.2cm}
  if  sign($\gamma_k) \neq$ sign$(\gamma_k - \alpha \lambda$ sign$(\gamma_k))$ then: $\gamma_k = 0$\\
\indent else $\gamma_k = \gamma_k - \alpha \lambda$ sign$(\gamma_k)$
\end{center}
This algorithm is called Iterative Shrinkage and Thresholding Algorithm:
\begin{algorithm}
 \caption{ISTA (Iterative Shrinkage and Thresholding Algorithm)}
 \begin{algorithmic}
  \STATE initialize $\gamma$ 
  \WHILE{$\gamma$ has not converged}
    \FORALL { $\gamma_k$ \textbf{in} $\gamma$ }
        \STATE $\gamma_k = \gamma_k - \alpha (D[:,k])^{\intercal}  (D\gamma - x)$
        \STATE $\gamma_k = shrink(\gamma_k,\alpha\lambda)$
        \ENDFOR
    \ENDWHILE
\RETURN $\gamma$
\STATE 
\STATE Here \textbf{shrink(a,b) }= [..., sign$(a_i)$ max($|a_i| - b_i$, 0), ...]

 \end{algorithmic}

\end{algorithm}
% 
% The original problem is a combinatorial problem (proven to be NP-hard). To solve this problem we use relaxation methods (then we can smooth the $L_0$ and use continuous optimization techniques) or greedy methods (then build the solution one non-zero element at time).
% \subsubsection{Compute $\alpha$ }
% %Inderence of sparse code=============================================
% \paragraph{Idea}f
% Here we develop  step (1) of our algorithm.\\
% Assume we are given a dictionary matrix D, how do we compute $h(x^{(t)})$. We have to optimize:
% \begin{center}  Basic Pursuit:
% $l(x^{(t)}) = \frac{1}{2} \| x^{t}- D \alpha^{(t)} \|^{2}_{2} + \lambda \|\alpha^{(t)}\|_1 w.r.t. \alpha^{(t)}$\\ 
% \end{center} 
% Here we used relaxation method to switch from norm $l_0$ to $l_1$ know as the  Basic Pursuit (vs Matching Pursuit, a greedy method,  if we keep $l_0$ norm and find one atom at a time).\\
% We could use a gradient descent method to solve this minimization:\\
% \begin{center}
% $\Delta_{\alpha^{(t)}} l(x^{(t)}) = D^T (D \alpha^{(t)} - x^{(t)}) + \lambda sign(\alpha^{(t)})$
% \end{center}
% The issue is $l_1$ norm is not differentiable at 0. The solution is : if $\alpha^{(t)}$ changes sign because of $l_1$ norm gradient then clamp to 0.That mean :
% 
% $\alpha^{(t)}_k = \alpha^{(t)}_k   - \alpha (D_{., k})^T (D \alpha^{(t)} - x^{(t)})$\\
% \indent if  sign($\alpha^{(t)}_k) \neq$ sign$(\alpha^{(t)}_k - \alpha \lambda$ sign$(\alpha^{(t)}_k) )$ then: $\alpha^{(t)}_k = 0$\\
% \indent else $\alpha^{(t)}_k = \alpha^{(t)}_k - \alpha \lambda$ sign$(\alpha^{(t)}_k)$
% \paragraph{ISTA (Iterative Shrinkage and Thresholding Algorithm)}
% :
% \begin{lstlisting}[language=Python,frame=single]
% initialize h 
% while h not_converged:
%     for each h_k in h:
%         h_k = h_k -alpha * transpose(D[:,k]) * (D*h - x)
%         h_k = shrink(h_k,alpha*lambda_coef)
% return h
% \end{lstlisting}
% Here \textbf{shrink(a,b) }= [..., sign$(a_i)$ max($|a_i| - b_i$, 0), ...]\\

\subsection{Application for MNIST dataset}
The MNIST database of handwritten digits, available from Yann Lecun's website. MNIST has a training set of 60,000 examples and a test set of 10,000 examples. It is a subset of a larger set available from NIST. The digits have been size-normalized and centred in a fixed-size image. In my test, I'll use 55000 examples from the training set (using Tensorflow datasets). These are $28 \times 28$ images. One way to evaluate the quality of our results is to compare the original data vs the reconstructed ones. 
\begin{figure}[h]
 \centering
 \includegraphics[scale=0.5]{MnistExamples.png}
 % MnistExamples.png: 594x361 px, 72dpi, 20.96x12.74 cm, bb=0 0 594 361
 \caption{Example of MNIST's handwitten digits}
\end{figure}
\subsubsection{Prototype}
My first task is to realise a Sparse Coding prototype to compute Sparse Coding on this dataset, using Python. The aim here is to understand the underlying principles of this method, you can found this prototype in \texttt{Code directory} of this repository as \texttt{SparseCoding.py}. \\
These are some results of this prototype: For time-saving, I used only 100 digits as input.

\subsubsection{SPAMS}
SPAMS (SPArse Modeling Software) is an optimization toolbox for solving various sparse estimation problems.
\begin{itemize}
 \item Dictionary learning and matrix factorization (NMF, sparse PCA, ...)
 \item Solving sparse decomposition problems with LARS, coordinate descent, OMP, SOMP, proximal methods
 \item Solving structured sparse decomposition problems (l1/l2, l1/linf, sparse group lasso, tree-structured regularization, structured sparsity with overlapping groups,...).
\end{itemize}
It is developed and maintained by Julien Mairal (Inria), and contains sparse estimation methods resulting from collaborations with various people: notably, Francis Bach, Jean Ponce, Guillermo Sapiro, Rodolphe Jenatton and Guillaume Obozinski.\\
You can find my code of Sparse Coding/Dictionary Learning on MNIST using SPAMS toolbox on my GitHub \texttt{test\_spams.py}.\\
\newpage
\paragraph{Test 1}
In the first test, I used 256 atoms, 2 000 iterations and $\lambda$ = 0.015 to learn the dictionary and the sparse coefficients. 
w\begin{figure}[h]
 \centering
 \includegraphics[scale=0.82]{../Results/SPAMS_X_ALL_K256/D.png}
 % D.png: 434x648 px, 100dpi, 11.02x16.46 cm, bb=0 0 312 467
 \caption{Few atoms of D}
\end{figure}

 \begin{figure}[h]
 \begin{subfigure}{.5\textwidth}
 \centering
 \includegraphics[scale=0.35]{../Results/SPAMS_X_ALL_K256/recons_1.png}
  \caption{Reconstructed 1 vs Original 1}
 % module-capteur-laser.jpg: 600x600 px, 72dpi, 21.17x21.17 cm, bb=0 0 600 600
 \end{subfigure}%
  \begin{subfigure}{.3\textwidth}
 \centering
 \includegraphics[scale=0.35]{../Results/SPAMS_X_ALL_K256/recons_3.png}
 % module-capteur-laser.jpg: 600x600 px, 72dpi, 21.17x21.17 cm, bb=0 0 600 600
  \caption{Reconstructed 3 vs Original 3}

 \end{subfigure}%
\end{figure}

\newpage

\paragraph{Test 2}In the second test I used 1024 atoms, 1 000 iterations and $\lambda = \frac{1.2}{\sqrt{m}}$ \cite{Mairal:2009:ODL:1553374.1553463} \textit{(In my case $\approx 0.0042857$)}.
\begin{figure}[h]
 \centering
 \includegraphics[scale=0.82]{../Results/SPAMS_X_ALL_K1024/D.png}
 % D.png: 1873x1022 px, 100dpi, 47.57x25.96 cm, bb=0 0 1349 736
 \caption{Few atoms of D}
\end{figure}
 \begin{figure}[h]
 \begin{subfigure}{.5\textwidth}
 \centering
 \includegraphics[scale=0.4]{../Results/SPAMS_X_ALL_K1024/lambdaopti_recons1.png}
  \caption{Reconstructed 1 vs Original 1}
 % module-capteur-laser.jpg: 600x600 px, 72dpi, 21.17x21.17 cm, bb=0 0 600 600
 \end{subfigure}%
  \begin{subfigure}{.3\textwidth}
 \centering
 \includegraphics[scale=0.29]{../Results/SPAMS_X_ALL_K1024/lambdaopti_recons3.png}
 % module-capteur-laser.jpg: 600x600 px, 72dpi, 21.17x21.17 cm, bb=0 0 600 600
  \caption{Reconstructed 3 vs Original 3}

 \end{subfigure}%
\end{figure}
\newpage
\paragraph{Test 3} In the third test I used 1024 atoms, 1 000 iterations and $\lambda = 5$.
\begin{figure}[h]
 \centering
 \includegraphics[scale=0.82]{../Results/SPAMS_X_ALL_K1024/D_lambdagrand.png}
 % D_lambdagrand.png: 936x994 px, 100dpi, 23.77x25.25 cm, bb=0 0 674 716
 \caption{Few atoms of D}
\end{figure}

 \begin{figure}[h]
 \begin{subfigure}{.5\textwidth}
 \centering
 \includegraphics[scale=0.4]{../Results/SPAMS_X_ALL_K1024/lambdagrand_recons1.png}
  \caption{Reconstructed 1 vs Original 1}
 % module-capteur-laser.jpg: 600x600 px, 72dpi, 21.17x21.17 cm, bb=0 0 600 600
 \end{subfigure}%
  \begin{subfigure}{.3\textwidth}
 \centering
 \includegraphics[scale=0.39]{../Results/SPAMS_X_ALL_K1024/lambdagrand_recons3.png}
 % module-capteur-laser.jpg: 600x600 px, 72dpi, 21.17x21.17 cm, bb=0 0 600 600
  \caption{Reconstructed 3 vs Original 3}

 \end{subfigure}%
\end{figure}

\newpage
\paragraph{Test 4} In the fourth test I used 2048 atoms, 1 000 iterations and $\lambda = \frac{1.2}{\sqrt{m}}$
\begin{figure}[h]
 \centering
 \includegraphics[scale=0.82]{../Results/SPAMS_X_ALL_K_2048/D_K2048.png}
 % D_K2048.png: 936x994 px, 100dpi, 23.77x25.25 cm, bb=0 0 674 716
 \caption{Few atoms of D}
 \end{figure}
 
  \begin{figure}[h]
 \begin{subfigure}{.5\textwidth}
 \centering
 \includegraphics[scale=0.4]{../Results/SPAMS_X_ALL_K_2048/recons1.png}
  \caption{Reconstructed 1 vs Original 1}
 % module-capteur-laser.jpg: 600x600 px, 72dpi, 21.17x21.17 cm, bb=0 0 600 600
 \end{subfigure}%
  \begin{subfigure}{.3\textwidth}
 \centering
 \includegraphics[scale=0.39]{../Results/SPAMS_X_ALL_K_2048/recons3.png}
 % module-capteur-laser.jpg: 600x600 px, 72dpi, 21.17x21.17 cm, bb=0 0 600 600
  \caption{Reconstructed 3 vs Original 3}

 \end{subfigure}%
\end{figure}

\newpage

\subsection{Application for Lenna}

\begin{figure}[h]
 \centering
 \includegraphics[scale=0.3]{../Results/Lenna/lenna_1024.png}
 % lenna_1024.png: 936x994 px, 100dpi, 23.77x25.25 cm, bb=0 0 674 716
 \caption{Some atoms of D when K = 1024}
\end{figure}
\begin{figure}[h]
 \centering
 \includegraphics[scale=0.75]{../Results/Lenna/lenna_2048.png}
 % lenna_2048.png: 640x480 px, 100dpi, 16.26x12.19 cm, bb=0 0 461 346
 \caption{Some atoms of D when K = 2048}
\end{figure}
\newpage

\subsection{Dictionary learning and sparse coding for unsupervised clustering}
\label{sec:Clustering}
Whereas the previous tests seem to have a good result, one question appears \\ \textit{What makes us confident about the fact that two close images (two handwritten 3 for example) have close coefficients representation h ?}\\
Sprechmann and Sapiro \cite{5494985} propose an algorithm to cluster datasets that are well represented in the sparse modelling framework with a set of K learned dictionaries. The main idea is, given a set of K dictionaries, find for each signal in the dictionary for which the "best" sparse decomposition is obtained, with :

\begin{center}
$\underset{D_i,C_i}{\min} \sum_{i=1}^{K} \sum_{x_j \in C_i} \mathcal{R}(x_j, D_i)$
 
\end{center}
Here $D_i \in \R^{n \times k_i}$ is the $k_i$ dictionary associated with the class $C_i$. $x_j \in \R^n$ are the input data and $\mathcal{R}$  a function that mesure how good the sparse decomposition is for the signal $x_j$ under the dictionary $D_i$. Sprechmann and Sapiro propose to use the cost function in the Lasso-type problem as $\mathcal{R}$ the measure of performance, $\mathcal{R}(x,D) = \|x - D\alpha\|^2_2 + \lambda \|\alpha\|_1$. The class $\mathcal{C}$ for a given signal x is found by solving $\mathcal{C}= \underset{j=1,..,K}{\argmin}$ $ \mathcal{R}(x,Dj) $.

\subsubsection{Dictionary learning for clustering}
Given a set of signals and number of classes, we want to find a set of K learned dictionaries that best represent x (the input data). \cite{5494985} formulate thus as an energy minimization problem and use the measure previously proposed,\\
\begin{center}
 $\underset{D_i,C_i}{\min} \sum_{i=1}^{K} \sum_{x_j \in C_i} \underset{\alpha_{ij}}{\min}\|x_j - D_i \alpha_{ij}\|^2_2 + \lambda\|\alpha_{ij}\|_1$
\end{center}
The optimization is carried out by solving one problem at time:
\begin{itemize}
 \item \textit{Assignement step:} The dictionaries are fixed and each signals is assigned to the cluster for which the best representation is obtained.
 \item \textit{Update step:} The new dictionaries are computed fixing the assignation found in the previous step.
\end{itemize}
One drawback of this algorithm is there is no guarantee of reach a global minimum. In this setting, repeated initialization are computationally expensive, thus we need a good initialization.

\subsubsection{Initialization}
The initialization can be given by a set of K dictionaries or as an initial partition of the data.\\
The main idea is to construct a similarity matrix and use it as the input for a spectral clustering algorithm. Let define $A = [\alpha_1,....,  \alpha_m]$ with $\alpha_j$ the sparse representation of each signal $x_j$. To obtain a good classification, we expect two signal to the same cluster to have decomposition that uses similar atoms. Thus we can compute two similarity matrix:
\begin{itemize}
 \item \textit{Clustering the signals :} Construct a similarity matrix $S_1 \in \R^{m \times m}$ which measure the similarity of two signals by comparing the corresponding sparse representation:\\
 $S_1 = |A|^T |A|$
 \item \textit{Clustering the atoms :} Construct a similarity matrix $S_2 \in \R^{k_0 \times k_0}$ ( with $D_0 \in \R^{n \times k_0}$) which represent the similarity of two atoms by comparing how many signals use them simultaneously and how they contribute in their sparse decomposition.\\
 $S_2 = |A||A|^T$
\end{itemize}
In this two case, the similarity matrixes are positive semidefinite and can be associated with a graph: $G_1 = \{X,S_1\}$ and $G_2 = \{D,S_2\}$ where the data (respectivly atoms) are the sets of vertexes with the corresponding $S_i$ as edge weights matrixes. This graph is partitioned using standard spectral clustering algorithm to obtain the initialization.\\
However, when K is large (the number of class), the performance of initial clusterization decreases. To fix this problem \cite{5494985} proposed to stat with the whole set as the only partition and at each itation we subdivise in two sets each of the current partitions, the procedure stops when the desired number of clusters is reached.


\newpage
\section{Discriminative Dictionary }
There are serveral approach for discriminative Dictionary Learning enumerate by \cite{8294264}:
\begin{itemize}
 \item In presence of label:
    \begin{enumerate}
     \item Learn one dictionary per class
     \item Prune large dictionaries
     \item Jointly learn dictionary and classifier
     \item Embed class label into the learning of sparse coefficients
     \item Learn a histogram of dictionary element over signal constituents
    \end{enumerate}

 \item For weakly supervised:\\
    Serveral max margin based, non-convolutive, syntgesis dictionary learning approaches.
\end{itemize}
However in our problem of extract new features for speech we must compare each $\alpha$ with other, thus we must use only one dictionary.

\subsection{One dictionary per class}
The idea is to extend proposed method in  section \ref{sec:Clustering} by adding a term $\mathcal{Q}(D_i,D_j)$ that promotes incoherence between the differents dictionaries, e.g. this term encourage dictionaries to  be independent as possible :
\begin{center}
 $\underset{D_i, C_i}{\min} \sum_{i=1}^{K} \sum_{x_j \in C_i} \mathcal{R}(x_j,D_i) + \eta \sum_{i \neq j} \mathcal{Q}(D_i,D_j)$
\end{center}
For example we can take $\mathcal{Q}(D_i,D_j) = \|D^T_iD_j\|^2_F $ with F denotes Frobenius norm.

\subsection{Supervised Dictionary learning SDL}
\subsubsection{Problem formulation}
In \cite{mairal:inria-00322431} the signal may belong to any of $p$ diffrent classes and they model the signal using a single shared D. They create a set of $p$ decision functions $g_i(x,\alpha,\theta)$ (i = 1,...,p). Where: \\
\begin{center}
  $g_i(x,\alpha,\theta) =     \left\{
                \begin{array}{ll}
                 gi > 0 $ if $x \in $ class i $\\
                g_i \leq 0$ otherwise$\\
                \end{array}
              \right.$
\end{center}
The vector $\theta$ parametrizes the model and will be jointly learned with the dictionary D. There are two kinds of models in this paper:\\
\begin{itemize}
 \item \underbar{Linear in $\alpha$:} $g_i(x,\alpha,\theta) = w_i^T \alpha + b_i $ where $ \theta = \{ w_i \in \R^k, b_i \in \R\}_{i=1}^p$
 \item \underbar{Bilinear in $x$ and $\alpha$: }$g_i(x,\alpha,\theta) = x^T W_i \alpha + b_i$ where $ \theta = \{W_i \in \R^{n \times k}, b_i \in \R\}_{i=1}^p$ 
\end{itemize}
They define a \textit{softmax} discriminative cost function as :
\begin{center}
 $\mathcal{C}_i(x_1,..., x_p) = log(\sum_{i=1}^p e^{x_j - x_i})$
\end{center}
Given x, a input signal, with $D$ and $\theta$ fixed, the supervised sparse coding problem for the class $p$ can be computing by :
\begin{center}
 $S^*_i(x,D,\theta) = \underset{\alpha}{\min}S_i(\alpha,x,D,\theta)$
\end{center}
where
\begin{center}
 $S_i(\alpha,x,D,\theta) = \mathcal{C}_i(\{g_j(x,\alpha,\theta)\}^p_{j=1}) + \lambda_0 \|x - D\alpha\|^2_2 + \lambda_1 \|\alpha\|_1$
\end{center}
Then, the classification problem can be compute by: 
\begin{center}
 
$i^*(x,D,\theta) = \underset{i=1,...,p}{\argmin}S^*_i(x,D,\theta)$
\end{center}

\subsubsection{Learning $D$ and $\theta$}
The most direct method for learning $D$ and $\theta$ is to minimize with respect to these (with $T_i$ a sample of input signals corresponding to the class i):
\begin{center}
 $\underset{D,\theta}{\min}(\sum_{i=1}^p \sum_{j \in T_i}S^*_i(x_j,D,\theta))+ \lambda_2 \|\theta\|^2_2$
\end{center}
With  $ \|\theta\|^2_2$ to prevent overfitting. They reffer to this model as SDL-G (Supervised Dictionary Learning - Generative) \cite{mairal:inria-00322431}.\\
A more discriminative approach is not only make $S^*_i$ small for signals with label i but also make the value of $S^*_j$ (with $i \neq j)$ greater than $S^*_i$. To do that they use the softmax cost function $\mathcal{C}_i$:
\begin{center}
  $\underset{D,\theta}{\min}(\sum_{i=1}^p \sum_{j \in T_i} \mathcal{C}_i(\{ S^*_l(x_j,D,\theta)\}^{p}_{l=1}))+ \lambda_2 \|\theta\|^2_2$
\end{center}
But this more difficult to solve, thus they adopt a mixed formulation with SDL-G :
\begin{center}
   $\underset{D,\theta}{\min}(\sum_{i=1}^p \sum_{j \in T_i} \mu \mathcal{C}_i(\{ S^*_l(x_j,D,\theta)\}^{p}_{l=1}) + (1-\mu)S^*_i(x_j,D,\theta))+ \lambda_2 \|\theta\|^2_2$
\end{center}
They refer to this model as SDL-D (Supervised Dictionary Learning - Discriminative). With $\mu$ which control the trade-off between reconstruction and discrimination.

\subsubsection{Optimization  procedure}
\paragraph{SDL-G}
When $\mu = 0$ or directly SDL-G have the same properties than classical dictionary learning techniques: Using block coordinate descent  consist of iterating between \textit{supervised sparse coding}, where $D$ and $\theta$ are fixed and optimize $\alpha$, and \textit{supervised dictionary update}, where  $\alpha$ is fixed but $D$ and $\theta$ are updated.
\paragraph{SDL-D} However the discriminative version of SDL (where $\mu \neq 0$) is not convex (even when $D$ and $\theta$ or $\alpha$ are fixed). To reach a local minimum for this problem, they have chosen a continuation method: Starting from the generative case and ending with the discriminative one.
\begin{figure}[h]
 \centering
 \includegraphics[scale=0.9]{SDL-D.png}
 % SDL-D.png: 532x215 px, 96dpi, 14.07x5.69 cm, bb=0 0 399 161
 \caption{SDL: Supervised dictionary learning Algorithm  \cite{mairal:inria-00322431}}
\end{figure}



\section{Label Consistent K-SVD}
\cite{6516503} proposed a supervised learning algorithm to learn a compact and discriminative dictionary for sparse coding. They add a new label consistency constraint ( ``discriminative sparse-code error") and combining reconstruction error and classification error to form a unified objective function that can be solved with the K-SVD algorithm.
%\subsection{Dictionary Learning for classification}
%A good classifier $f(x)$ can be obtained by determining its model parameters $W \in R^{m \times K} $ satisfying:
%\begin{center}
% $W = \underset{W}{\argmin} \underset{i}{\sum}\mathcal{L}\{h_i,f(x_i,W)\} + \lambda_1 \|W\|^2_F$
%\end{center}
%where: $m$ is the number of classes, $\mathcal{L}$ is the classification loss function (typically logistic loss function, suqare hinge loss, simple quadratic loss function), $h_i$ the label of $y_i$ and $\lambda_1$ a regularization parameters (which prevents overfitting).\\
%The objective function for learning $D$ and $W$ jointly can be :
%\begin{center}
 %$<D,W,\alpha> = \underset{D,W,\alpha}{\argmin} \|X -D\alpha\|^2_2 + \underset{i}{\sum}\mathcal{L}\{h_i,f(\alpha_i,W)\} + \lambda_1 \|W\|^2_F$  s.t.  $\forall i, \|\alpha_i\|_0 \leq T$ 
%\end{center}

%The sparse code $\alpha_i$ can be used as a feature descriptor of input signal $x_i$, then the risk minimization formulation can be written as:
%\begin{center}
 %$<D,W> = \underset{D,W}{\argmin}\underset{i}{\sum}\mathcal{L}\{h_i,f(\alpha^*(x_i,D),W)\} + \frac{\nu_1}{2}\|W\|^2_F$
%\end{center}
%Here D is not explicitly defined in the function but implicitlu in the sparse coding step ($\alpha^*(x_i,D)$ )

\subsection{Idea}
In a supervised setting, during the training phase, all the information about an input signal is given, including its label. The idea is to use this information to learn a discriminative dictionary. To do this, \cite{6516503} add a new consistency regularization term, a classification error and label consistency regularization term into the objective function:
\begin{center}
 $<D,\gamma> = \underset{D,\gamma}{\argmin} \|X - D\gamma\|^2_2 + \lambda \|\gamma\|_0 +\underset{Label\ consistent\ term}{\underbrace{\mu \|Q -AX\|_2^2 }} + \underset{Consistency\ term}{\underbrace{\beta\|H-WH\|_2^2}}$
\end{center}
%They refer to this two problem as LC-KSVD1 and LC-KSVD2 respectivly.
Where Q is the discriminative sparse codes of input signals X for classification, A is a linear transformation matrix, W is the classifier parameters and H is the labels corresponding to the input signal X.\\
This optimization function is called LC-KSVD2. But there is a variant of this method, called LC-KSVD1 which is the same function without the consistency term.\\
In our setting, we will use LC-KSVD1 because we do not want to train a classifier at this stage.
\subsection{LC-KSVD1}
Therefore, the objective function for dictionary learning is  defined as:
\begin{center}
 $<D,A,\gamma> = \underset{D,A,\gamma}{\argmin} \| X - D\gamma \|^2_2 + \mu \|Q -A\gamma\|^2_2 + \lambda\|\gamma\|_0$
\end{center}

Where $\mu$ controls the contribution between reconstruction and label consistency regularization. $Q = [q_1,...,q_N] \in \R^{K \times N}$.\\
Q is a discriminative sparse code corresponding to an input signal and the dictionary atoms, the nonzero values occur when $signal_i$ and $atom_i$ share the same label (figure \ref{fig:Q}).\\
Q is chosen by the user, in our setting we have chosen to match the same number of atoms for each class. It is then possible to have "empty" atoms that are not discriminative (white color's atom in figure \ref{fig:Q}).
\begin{figure}[h]
 \centering
 \includegraphics[scale=0.7]{LCKSVDTable.png}
 % Q_explications.png: 487x177 px, 96dpi, 12.88x4.68 cm, bb=0 0 365 133
 \caption{Example of Q, each color corresponding to a class (white color is the lack of classes). Signal 1,3 and 6 are from class 1; signal 2 and 5 from class 2 and signal 5 from class 3}
  \label{fig:Q}

\end{figure}
To use K-SVD algorithm it is necessary to  rewrite the previous equation:
\begin{center}
$ <D,A,\gamma> = \underset{D,A,\gamma}{\argmin} \| \begin{pmatrix} 
X  \\
\sqrt{\mu}Q
\end{pmatrix} -  \begin{pmatrix} 
D  \\
\sqrt{\mu}A
\end{pmatrix}\gamma \|^2_2\lambda \|\gamma\|_0$
\end{center}
We define $X_{new}  =  \begin{pmatrix} 
X  \\
\sqrt{\mu}Q
\end{pmatrix}$ and $D_{new} =  \begin{pmatrix} 
D  \\
\sqrt{\mu}A
\end{pmatrix}$. \\ \hspace{0.5cm}
It's important to note that the $D_{new}$ matrix is $L_2$ normalized. Now, it is possible to solve the minimization problem using the K-SVD algorithm:
\begin{center}
 $<D,A,\gamma> = \underset{D,A,\gamma}{\argmin} \| X_{new} - D_{new}\gamma \|^2_2 + \lambda \|\gamma\|_0$
\end{center}
However, it is impossible to use directly D and A  after employing the K-SVD algorithm. It is because D and A are $L_2$ normalized due to $D_{new}$. The desired $\hat{D}$ and $\hat{A}$ can be computed as follows:\vspace{0.5cm}\begin{center}
$\hat{D} = (
\frac{d_1}{\|d1\|_2}  ...  \frac{d_K}{\|d_K\|_2} 
)$   and $\hat{A} =
(\frac{a_1}{\|d_1\|_2}  ...  \frac{a_K}{\|d_K\|_2} 
)$
\end{center}
Label-Consistent K-SVD algorithm is given as follow:
\begin{algorithm}
 \caption{Label Consistent K-SVD 1 (LC-KSVD1)}
 \begin{algorithmic}
  \REQUIRE $X, Q, \lambda_1$
  \STATE $D_0$ = KSVD($X, \lambda_1$)
  \STATE $A_0 = Q  X^{\intercal} (X  X^{\intercal} + \lambda_2  I)^{-1}$
  \STATE $\gamma_0$ = OMP($D_0, X, lambda_1$)
  \STATE $X_{new}  =  \begin{pmatrix} 
    X  \\
    \sqrt{\mu}Q
    \end{pmatrix}$
  \STATE $D_{new} =  \begin{pmatrix} 
    D  \\
    \sqrt{\mu}A
    \end{pmatrix}$
  \STATE $ D = $ KSVD($D_{new}, X_{new}, \lambda_1)$
  \STATE $\hat{D} = (
\frac{d_1}{\|d1\|_2}  ...  \frac{d_K}{\|d_K\|_2} 
)$
\STATE $\hat{A} =
(\frac{a_1}{\|d_1\|_2}  ...  \frac{a_K}{\|d_K\|_2} 
)$
\RETURN $\hat{D}, \hat{A}$
 \end{algorithmic}

\end{algorithm}

%\vspace{0.5cm}
%\begin{lstlisting}[language=Python,frame=single]
%Input: X, Q, lambda1, K
%Output: D, A
%===========================================================================
%D_0 = K-SVD(X,lambda1)
%A_0 = Q * tranpose(X) * inv(X*tranpose(X) + lamba2 * Id)
%h_0 = omp(Dnew,Xnew,lambda1)

%Initialize Xnew and Dnew
%D = K-SVD(Dnew,Xnew,lambda1)

%Extract D, A
%return D, A
%\end{lstlisting}
%\vspace{0.5cm}
%With $A_0 = Q X^T (XX^T + \lambda_2 I)^{-1}$
%\vspace{0.5cm}\\
%In our case we don't want to train the classifier, we only want to find new features. This is why we only apply LC-KSVD1.\\
%You can find my Python code of  LC-KSVD1 on MNIST in my Github repository: \texttt{LC-KSVD.py}

\newpage

\subsection{Application on MNIST}
For k = 1024 and $\mu = 5$:

\begin{figure}[h]
 \centering
 \includegraphics[scale=0.5]{../Results/LC-KSVD_X_ALL_K_1024/D.png}
 % D.png: 1366x660 px, 100dpi, 34.70x16.76 cm, bb=0 0 984 475
 \caption{Examples of D's atoms}
\end{figure}

\begin{figure}[h]
 \centering
 \includegraphics[scale=0.5]{../Results/LC-KSVD_X_ALL_K_1024/repartition_h.png}
 % répartition_h.png: 681x651 px, 100dpi, 17.30x16.54 cm, bb=0 0 490 469
 \caption{Sparse coefficients repartition}
\end{figure}
\begin{figure}[h]
 \centering
 \includegraphics[scale=0.72]{../Results/LC-KSVD_X_ALL_K_1024/confusion_matrix_train.png}
 % confusion_matrix_train.png: 640x480 px, 100dpi, 16.26x12.19 cm, bb=0 0 461 346
 \caption{Kmeans classification on A$\alpha$ from the trainning dataset}
\end{figure}


 \begin{figure}[h]
 \begin{subfigure}{.5\textwidth}
 \centering
 \includegraphics[scale=0.29]{../Results/LC-KSVD_X_ALL_K_1024/3_recons.png}
  \caption{Reconstructed 3 vs Original 3}
 % module-capteur-laser.jpg: 600x600 px, 72dpi, 21.17x21.17 cm, bb=0 0 600 600
 \end{subfigure}%
  \begin{subfigure}{.3\textwidth}
 \centering
 \includegraphics[scale=0.29]{../Results/LC-KSVD_X_ALL_K_1024/8_recons.png}
 % module-capteur-laser.jpg: 600x600 px, 72dpi, 21.17x21.17 cm, bb=0 0 600 600
  \caption{Reconstructed 8 vs Original }

 \end{subfigure}%
\end{figure}

\begin{figure}[h!]
 \centering
 \includegraphics[scale=0.72]{../Results/LC-KSVD_X_ALL_K_1024/confusion_matrix_test.png}
 % confusion_matrix_test.png: 640x480 px, 100dpi, 16.26x12.19 cm, bb=0 0 461 346
 \caption{Kmeans classification on A $\alpha$ from the test dataset w}
\end{figure}

\newpage
\newpage

\subsection{Application on ``voyelle'' dataset }
We have for our experiments some audio signals that have been recorded in a studio. The speaker pronounces 10 vowels. There are 100 occurrences of each vowel (\textit{aa, ee, eh, eu, ii, oe, oh, oo, uu, yy}). Each signal is of 1024 sample.\\
A cepstral parametrization has been extracted from these samples, and a PCA has been done to reduce the data's dimension to  2 and 12. \vspace{0.5cm} \\
First, we focus our research on 2 dimension dataset. There are 10 classes which correspond to 10 vowels.\newline
\begin{figure}[h]
 \centering
 \includegraphics[scale=1]{../Results/Voyelles/voyelle_2.png}
 % voyelle_2.png: 445x648 px, 100dpi, 11.30x16.46 cm, bb=0 0 320 467
 \caption{Kmean classification on A$\alpha$ from the train dataset}
\end{figure}
 \\
\textit{Note: Here there is bad cluster identification (this is why we don't have a perfect diagonal matrix), however, the result is the same: In this case, our method is bad for classification because on the same row or column there is sometimes a bad repartition. It's due to the length of our input vector (2 dimensional vector), which is not enough to get the good result for classification using this method of Sparse representation of the signal.} \vspace{0.5cm}
\newpage
Then we focus our approach on 12 dimension dataset:
\begin{figure}[h]
 \centering
 \includegraphics[scale=0.49]{../Results/Voyelles/voyelle_13_train.png}
 % voyelle_13_train.png: 800x630 px, 100dpi, 20.32x16.00 cm, bb=0 0 576 454
 \caption{Kmean on data12 train set}
\end{figure}
\begin{figure}[h]
 \centering
 \includegraphics[scale=0.5]{../Results/Voyelles/voyelle_13_test.png}
 % voyelle_13_test.png: 1366x660 px, 100dpi, 34.70x16.76 cm, bb=0 0 984 475
 \caption{Kmean on data12 test set}
\end{figure}
 \begin{figure}[h]
 \centering
 \includegraphics[scale=0.45]{../Results/Voyelles/VoyellesAhRepartition.png}
 % VoyellesAhRepartition.png: 1366x660 px, 100dpi, 34.70x16.76 cm, bb=0 0 984 475
 \caption{ Four examples of A$\alpha$ from two different classes (orange and blue : Class 1, red and green: Class 2)}
\end{figure}
\newpage
\subsection{Discussion}
LC-KSVD perform well on our two datasets (MNIST and voyelle) when the input data are large enough (when input equal 2 we can see we get a bad result of clusterization). With this method we have two dictionaries: 
\begin{enumerate}
 \item \textbf{D} : A reconstructive dictionary
 \item \textbf{A}: A discriminative dictionary
\end{enumerate}
Using \textbf{A} we get a new representation of our input, a discriminative one, that we can put in the input of classifier to get a good classification.\\
However, this method works well on non-shift variant signals. When we have shift variant signal this method will fail to get a good discriminative representation i.e. If we have the same signal but delay with $\delta$, LC-KSVD will fail to get a close representation of the two signals.

\newpage
\section{Convolutional Sparse Coding}
Notations:
\begin{itemize}
 \item Matrix: Uppercase  \textbf{A}
 \item Vector: Lower-case  \textbf{a}
 \item Scalar: Lower-case a
 \item 2D convolution operation : *
 \item $D \times D$ identity matrix: \textbf{$I_D$}
 \item Kronecker product : $\odot$
 \item $D \times D$ Fourier matrix : \textbf{F}
 \item Hadamard product: $\otimes$
 \item inverse FFT : $ \mathcal{F}^{-1}$
 \item Mapping function that preserve only $M \ll D$ active values relating to the small spaital structure of the estimated filter : $ \mathcal{M}\{\}$
\end{itemize} 

For the simplicity of reading, we will take some new notation convention during this chapter:\\
\begin{itemize}
 \item Matrix: Uppercase  \textbf{A}
 \item Vector: Lower-case  \textbf{a}
 \item Scalar: Lower-case a
 \item 2D convolution operation : *
 \item $D \times D$ identity matrix: \textbf{$I_D$}
 \item Kronecker product : $\odot$
 \item $D \times D$ Fourier matrix : \textbf{F}
 \item Hadamard product: $\otimes$
 \item inverse FFT : $ \mathcal{F}^{-1}$
 \item Mapping function that preserve only $M \ll D$ active values relating to the small spatial structure of the estimated filter : $ \mathcal{M}\{\}$
\end{itemize} 


\section{Idea}
\begin{figure}[h]
 \centering
 \includegraphics[scale=0.3]{csc_idea.png}
 % csc_idea.png: 926x551 px, 72dpi, 32.67x19.44 cm, bb=0 0 926 551
 \caption{Idea behind Convolutional Sparse Coding}
 \label{fig:csc_idea}
\end{figure}
In signal processing, shift variance is an important factor to consider. Indeed, there are no guarantees that the searched pattern is at the same location in the signal.\\
Based on Sparse Coding, other algorithms like Efficient Shift-Invariant Dictionary learning \cite{Zheng:2016:ESD:2939672.2939824} or Convolutional Sparse Coding \cite{6618901} refers to the problem of finding a set of latent basis vectors that capture informative \textit{local patterns} at different locations in the input sequence and not in the entire input sequence (example in figure \ref{fig:csc_idea}).\\
Among these two methods, we choose to use Convolutional Sparse coding because of the Shift-invariant power of convolution.

\subsection{Problem formulation}
\begin{figure}[h]
 \centering
 \includegraphics[scale=0.7]{CSC_superslide.png}
 % CSC_superslide.png: 662x351 px, 96dpi, 17.51x9.29 cm, bb=0 0 496 263
 \caption{CSC principle by Michael Elad \protect\footnotemark}
 \label{fig:CSC_slide}
\end{figure}
\footnotetext{\url{https://elad.cs.technion.ac.il/wp-content/uploads/2018/07/ICML_Talk_July_2018.pdf}}

Instead  of decomposing a signal as $x = D\gamma$, Convolutional Sparse Coding is the summation of convolutions between feature map $z$ (i.e. which refer to $\gamma$ in the traditional sparse coding) and the corresponding filter $d$ (i.e. our atoms in the dictionary in traditional Sparse Coding) in the dictionary $D$ (figures \ref{fig:CSC_slide} and \ref{fig:CSC_test}). So we have:\\
\begin{center}
 $X = \sum_{i=1}^{m} d_i * Z_i$\\  \vspace{0.4cm}
  $\underset{d, z}{\argmin}  \frac{1}{2} \| x - \sum_{k=1}^{K} d_k * z_k \|_2^2 + \beta \|z_k \|_1$\\

\end{center}
\begin{figure}[h]
 \centering
 \includegraphics[scale=1.5]{CSC_test.png}
 % CSC_test.png: 1038x468 px, 300dpi, 8.79x3.96 cm, bb=0 0 249 112
 \caption{Example of reconstruction using atoms $d_j$ and features map $z_j$\protect\footnotemark}
 \label{fig:CSC_test}
\end{figure}
\footnotetext{\url{http://jsulam.cswp.cs.technion.ac.il/wp-content/uploads/sites/27/2017/12/Presentation_Handout.pdf}}
This new approach assumes that all input vectors X are independent of each other. But the complexity of convergence dramatically increases.
Bristow \cite{6618901} proposed an Augmented Lagrangian and the use of  Alternative Direction Method of Multipliers (ADMM) to solve this problem. His approach is based on three key points:
\begin{itemize}
 \item The use of ADMM
 \item The convolution subproblem can be solved efficiently (and explicitly) in the Fourier domain (instead of temporal domain)
\item The use of quad-decomposition of the objective into four subproblems (which are convex)
\end{itemize}
In this approach, to solved a Convolutional Sparse Coding problem they introduced two auxiliary variables: \textbf{t} and \textbf{s} and posing the objective in the Fourier domain:
\begin{center}
$\underset{d, s, z, t}{\argmin} \hspace{0.4cm} \frac{1}{2D} \| \hat{x} - \sum_{k=1}^{K} \hat{d}_{k} \odot \hat{z}_k \|^2_2 + \beta \sum_{k=1}^{K} \|t_k\|_1$\\
subject to $\hspace{0.4cm} \|s_k\|^2_2 \leq 1 $ for k = 1.. K\\
                $ \hspace{2.4cm}s_k = \Phi^T \hat{d}_k$ for k = 1..K\\
                $\hspace{1.9cm}z_k = t_k$ for k = 1..K
\end{center}
With $\Phi$ a D $\times$ M submatrix of the Fourier matrix $F = [\Phi,\Phi_\bot]$. In this formulation $\hat{d}_k$ is a D-dimensional vector whereas in the original formulation $d_k \in \R^M$  is of a significantly smaller dimensionality to $M \ll D$  corresponding to its smaller spatial support. 


\section{Inference of CSC}

\subsection{Augmented Lagrangian}
We handle the introduction of new equality constraints through an augmented Lagrangian approach \cite{6618901}:\\
\begin{center}
$ \hspace{-8cm} \mathcal{L}(d,s,z,t, \lambda_s, \lambda_t) =$ \\

$ \frac{1}{2D} \| \hat{x} - \sum_{k=1}^{K} \hat{d}_k \odot \hat{z}_k \|^2_2 + \beta \|t\|_1$\\
$ + \lambda^{T}_s(s-[\Phi^T \otimes I_K]\hat{d}) + \lambda^{T}_t (z - t)$\\
$ \hspace{-1.7cm}+ \frac{\mu_s}{2} \| s - [\Phi^T \otimes I_K]\hat{d}\|^2_2$\\
$ \hspace{-3.2cm}+ \frac{\mu_s}{2} \|z-t\|^2_2$
\end{center}
\subsection{Quad-decomposition of the objective}
We decompose our objective into four convex subproblems:


\subsubsection{Subproblem z}
$z^* = \underset{z}{\argmin}$ $\mathcal{L}(z,d,s,t, \lambda_s, \lambda_t)$\\
$ = \mathcal{F}^{-1}\{\underset{z}{\argmin} \frac{1}{2}\|\hat{x} - \hat{D} \hat{z}\| + \hat{\lambda}_t^T(\hat{z}  - \hat{t}) + \frac{\mu_t}{2}\|\hat{z} - \hat{t}\|^2_2\}$\\
$= \mathcal{F}^{-1}\{(\hat{D}^T \hat{D} + \mu_tI)^{-1}(\hat{D}^T \hat{x} +  \mu \hat{t} - \hat{\lambda}_t)\} \vspace{0.6cm}$\\
Where $\hat{D} = [diag(\hat{d}_1),...,diag(\hat{d}_K)]$


\subsubsection{Subproblem t}
$t^* =  \underset{t}{\argmin}$ $ \mathcal{L}(t,d,s,z,\lambda_s,\lambda_t)$\\
$ =  \underset{t}{\argmin}\frac{\mu_t}{2} \|z - t\|^2_2 + \lambda_t^T(z-t) + \beta\|t\|_1$\\
Unlike subproblem z, the solution to $t$ cannot be efficiently computed in the Fourier domain (since $L_1$ norm is not rotation invariant). Solving $t$ requires projecting $\hat{z}$ and $\hat{\lambda}_t$ back into the spatial domain. If this equation does not contain any rotations of the data, each element of $t$ can be solved independently:\vspace{0.3cm}\\
$t^* =  \underset{t}{\argmin} \beta |t| + \lambda_t(z-t) + \frac{\mu}{2}(z-t)^2$\\
Where the optimal value of each $t$ can be found using shrinkage function:\vspace{0.5cm}\\
$t^* = sign(z + \frac{\lambda_t}{\mu_t}) . max\{|z + \frac{\lambda_t}{\mu_t} | - t , 0\}$

\subsubsection{Subproblem d}
$d^* =  \underset{s}{\argmin}$ $  \mathcal{L}(d,s,z,t,\lambda_s, \lambda_t)$\\ 
$ = \mathcal{F}^{-1} \{ \underset{\hat{d}}{\argmin}\frac{1}{2} \|\hat{x} - \hat{Z}\hat{d}\|^2_2 + \hat{\lambda}_s^T(\hat{d}- \hat{s}) + \frac{\mu}{2}\|\hat{d} - \hat{s}\|^2_2\}$\\
$ = \mathcal{F}^{-1}\{(\hat{Z}^T \hat{Z} + \mu_s I)^{-1} (\hat{Z}^T \hat{x} + \mu_s \hat{s} - \hat{\lambda}_s)\}$
\subsubsection{Subproblem s}
$s^* =\underset{s}{\argmin}$ $  \mathcal{L}(d,s,z,t,\lambda_s, \lambda_t) $\\
$  = \underset{s}{\argmin}$ $\frac{\mu_s}{2} \|\hat{d} - [\Phi^T \otimes I_K]s\|^2_2 + \hat{\lambda}_s^T(\hat{d} - [\Phi^T \otimes I_K]s)$\\
Solving this equation as it is a quadratically constrained quadratic programming problem (QCQP).  Due to the Kronecker product with the identity matrix $I_K$ it can be broken down into K independent problems:\\
$s^*_k = \underset{s_k}{\argmin}$ $\frac{\mu_s}{2} \|\hat{d}_k  - \Phi^T s_k\|^2_2 + \hat{\lambda}_{sk}^T(\hat{d}_k - \Phi^T s_k)$\\
Further, since $\Phi$ is orthonormal projecting the optimal solution to the unconstrained problem cab be found efficiently through:\\

\begin{center}
  $s^* =     \left\{
                \begin{array}{ll}
                 \|\tilde{s}\|^{-1}_2 \tilde{s}_k $ , if $ \|\tilde{s}\|^{-1}_2 \geq 1\\
                \tilde{s}_k $ otherwise$\\
                \end{array}
              \right.$
\end{center}
where,
\begin{center}
 $\tilde{s}_k = (\mu_s \Phi \Phi^T)^{-1} (\Phi \hat{d}_k + \Phi \hat{\lambda}_{sk})$
\end{center}

Finally, the solution to this equation can be found using :\\
$\tilde{s}_k = \mathcal{M}\{\frac{1}{\mu_s} \sqrt{D}^{-1}(  \mathcal{F}^{-1}\{\hat{d}_k\} +  \mathcal{F}^{-1}\{\hat{\lambda}_{sk}\}  ) \}$\\


\subsection{Lagrange Multiplier Update}
\begin{center}
 $\lambda^{(i+1)}_{t} \leftarrow \lambda^{(i)}_{t}  + \mu_t(z^{(i+1) - t^{(i+1)}}) $\\
 $\lambda^{(i+1)}_{s} \leftarrow \lambda^{(i+1)}_{s}  +\mu_s(d^{(i+1) - s^{(i+1)}}) $\\
\end{center}

\subsection{Penalty update}
Convergence may be reached if $\mu^{(i)} \rightarrow \infty$:

\begin{center}
  $\mu^{(i+1)} =     \left\{
                \begin{array}{ll}
                 \tau \mu^{(i)}$ if $\mu^{(i)} < \mu_{max} \\
                \mu^{(i)}$ otherwise$\\
                \end{array}
              \right.$
\end{center}

\subsection{SPORCO}
SPORCO (SParse Optimization Research COde) is a Python package for solving optimization problem with sparsity problem. These conssit primarily of sparse coding and dictionary learning problems but there is also support for other problems suchs as Convolutional sparse coding \cite{wohlberg-2017-sporco, wohlberg-2016-sporco}

You can find my code of Convolutional Sparse Coding/Dictionary learning on MNIST using SPORCO toolbox on my github \texttt{CSC.py}.\\

\newpage
\subsection{Application on random images}
In this section we show result for applying unsupervised Convolutional Sparse Coding (denoted as CSC) presented before on 7 random images, 6 for our train and 1 for our test.
\begin{figure}[h]
 \centering
 \includegraphics[scale=0.6]{../Results/SPORCO_test_img/Figure_2.png}
 % Figure_2.png: 663x630 px, 100dpi, 16.84x16.00 cm, bb=0 0 477 454
 \caption{Dictionary learned with 6 random images}
\end{figure}
\begin{figure}[h]
 \centering
 \includegraphics[scale=0.455]{../Results/SPORCO_test_img/activations.png}
 % activations.png: 663x630 px, 100dpi, 16.84x16.00 cm, bb=0 0 477 454
 \caption{Activation}
\end{figure}
\begin{figure}[h]
 \centering
 \includegraphics[scale=0.4]{../Results/SPORCO_test_img/recons.png}
 % recons.png: 795x384 px, 72dpi, 28.05x13.55 cm, bb=0 0 795 384
 \caption{Original image vs Reconstructed one}
\end{figure}


\newpage

\subsection{Application on MNIST}
In this section we show result for applying unsupervised Convolutional Sparse Coding (denoted as CSC) presented before on MNIST dataset.
\begin{figure}[h]
 \centering
 \includegraphics[scale=0.6]{../Results/SPORCO_MNIST/D.png}
 % D.png: 445x648 px, 100dpi, 11.30x16.46 cm, bb=0 0 320 467
 \caption{Filters from D}
\end{figure}
\begin{figure}[h]
 \centering
 \includegraphics[scale=0.6]{../Results/SPORCO_MNIST/activation.png}
 % CartesActivations.png: 663x648 px, 100dpi, 16.84x16.46 cm, bb=0 0 477 467
 \caption{Activation map}
\end{figure}



 \begin{figure}[h]
 \begin{subfigure}{.5\textwidth}
 \centering
 \includegraphics[scale=0.5]{../Results/SPORCO_MNIST/reconstruction.png}
  %\caption{Reconstructed 7 vs Original 1}
 % module-capteur-laser.jpg: 600x600 px, 72dpi, 21.17x21.17 cm, bb=0 0 600 600
 \end{subfigure}%
  \begin{subfigure}{.3\textwidth}
 \centering
 \includegraphics[scale=0.5]{../Results/SPORCO_MNIST/Original.png}
 % module-capteur-laser.jpg: 600x600 px, 72dpi, 21.17x21.17 cm, bb=0 0 600 600
  %\caption{Reconstructed 3 vs Original 3}

 \end{subfigure}%

\caption{Reconstructed 7 vs Original 7}
 \end{figure}


\newpage
\subsection{Discussion}
Using this examples we validate that CSC allow us to reconstruct our input data using filters, convolutions and activation maps. But we still cannot use activation maps for clustering or classification due to the non-discriminative reconstruction. Indeed, two seven in our test dataset haven't the same decomposition ( i.e. they don't have the same or close activation maps).

\newpage
\section{Supervised Convolutional Sparse Coding}
TODO
%====================PART 2 SPARSE CODING FOR SR=========
\newpage
\section{Sparse Coding for speech recognition}
In this part of the paper we will see novel feature exraction technique based on the principles of sparse coding \cite{DL_speech_reco}. Sparse codigin deals with the problem of how represent a given audio input as a linear combination of a minimum number of basis function. The weights of the linear combination are used as feature for speech recognition (acoustic modeling). Note the input dimensionality is typically \textbf{much} less than the number of atoms in the dictionary \textit{i.e.} we use overcomplete dictionary.\\
We use Sparse Coding algorithm as describe before and we get the dictionary D and  the matrix of sparse coefficients  h.\\
\paragraph{Reflection path} In \cite{DL_speech_reco} they used spectro-temporal speech domain wich is obtained by performing a short time Fourier transform (STFT) with an analysis window of length 25 ms and a frameshit of 10 ms on the input signal. Log critical band energies are subsequently obtained by projecting the magnitude square values of the STFT output on a set of frequency weights, which are equally spaced on the Bark frequence scale, and then applying a logarithm on the output projections.
\newpage
\bibliographystyle{plain}
\bibliography{efficient-sparse-coding-algorithms.bib}

\end{document}
